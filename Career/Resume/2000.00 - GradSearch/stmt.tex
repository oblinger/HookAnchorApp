\documentstyle{article}
\begin{document}
\title{}
\author{Dan Oblinger}
\date{\today}

\begin{center}
{\LARGE\bf Statement of Purpose\\}
\vspace{.05in}
{\large Daniel Anthony Oblinger\\401-11-1846}
\end{center}
\vspace{.3in}
It is difficult to say when I began to study ``computer science.''  At the age of two, the only thing I really wanted for Christmas was a pair of extension cords.  
By the age of five I was wiring batteries in parallel and serial, to power lights, motors, or anything else which ran on electricity.
Ironically, because I am dyslexiac, my scholastic performance in primary school and junior high, did not reflect my aptitude or intellectual curiosity.

My first experience with computing, was in tenth grade, when, on paper, I wrote machine language programs inspired by a data book I dug up for the Intel 8080.
I was instantly smitten with this complex idea of computation.  Among many self-directed  projects, I developed and implemented a text editor/programming environment specifically for quadriplegic users - it's still in use, and I implemented an efficient, usable dialect of LISP.

As the tasks required of me in school became more abstract, dyslexia became less of an issue; until in tenth grade, I realized I had compensated for it, and could be making As in all subjects.  So I did.  
As a freshman in college I became interested in mathematics.  Through mathematics I realized that I have the ability to, and enjoy, finding {\em models} which provide leverage for the problem being considered.  
AI also attracted my attention because its goal is to find appropriate models for cognition.  
In general, my interest shifted from the implementation to the model used.  I designed/redesigned a frame based, object oriented language for home automation.
After consuming all relevant classes in computer science and mathematics, I petitioned for/organized several seminars and independent studies.

At OSU I have passed the qualifying exam for the PhD program, taken 23 hours of seminar work and enjoyed being an instructor for several classes (including a 400 level class).  
I also spent last summer as an intern at IBM.  Two threads have emerged from these experiences, I am certain that I want to obtain a tenure track position where my primary responsibility will be research.  
And, although I have several areas of interest, I am particularly drawn toward studies in the representation/use of knowledge, and how this fits in with the rest of AI.
My other interests are in domains primarily concerned with the construction of a formal model, for example axiomatic semantics, or learning.
%%%My professor, Tim Long, has suggested a handful of places, including Yale, that would be instrumental in fulfilling my ambition.  Of the places Tim mentioned Yale has the greatest appeal because I am familiar with the research of the faculty at Yale, in particular Roger Shank, and I think their approach to knowledge representation is interesting and promising.
%%%My professor, Tim Long, has suggested a handful of places, including Stanford, that would be instrumental in fulfilling my ambition.  Of the places Tim mentioned Stanford has great appeal because of the breadth of the department and because of the work being done in knowledge representation.
%%%My professor, Dr. Chandrasekaran, has suggested a handful of places, including MIT, that would be instrumental in fulfilling my ambition.  Of the places Chandra mentioned MIT has great appeal because of the breadth of the department and because of the work being done in knowledge representation.
%%%My professor, Dr. Chandrasekaran, has suggested a handful of places, including UCLA, that would be instrumental in fulfilling my ambition.  Of the places Chandra mentioned UCLA has great appeal because of the breadth of the department and because of the work being done in knowledge representation.
%%%My professor, Tim Long, has suggested a handful of places, including Cornell, that would be instrumental in fulfilling my ambition.  Of the places Tim mentioned Cornell has great appeal because of the breadth of the department and because of the work being done both AI and programming languages.
%%%My professor, Dr. Chandrasekaran, has suggested a handful of places, including the University of Michigan, that would be instrumental in fulfilling my ambition.  Of the places Chandra mentioned the University of Michigan has great appeal because of the breadth of the department and because of the work being done in AI and specifically in knowledge representation.
%%%My professor, Dr. Chandrasekaran, has suggested a handful of places, including the University of Illinois, that would be instrumental in fulfilling my ambition.  Of the places Chandra mentioned the University of Illinois has great appeal because of the breadth of the department and because of the work being done in both knowledge representation and learning.
%%%My professor, Dr. Chandrasekaran, has suggested a handful of places, including the University of Texas, that would be instrumental in fulfilling my ambition.  Of the places Chandra mentioned the University of Texas has great appeal because of the breadth of the department and because of the work being done in AI and specifically in knowledge representation.
%%%My professor, Dr. Chandrasekaran, has suggested a handful of places, including Berkeley, that would be instrumental in fulfilling my ambition.  Of the places Chandra mentioned Berkeley the has greatest appeal because I am familiar with the research of the faculty, in particular Bob Wilensky, and I think this approach to knowledge representation is interesting and promising.
My professor, Dr. Chandrasekaran, has suggested a handful of places, including Northwestern, that would be instrumental in fulfilling my ambition.  
More specifically, Professor Roger Shank will probably be part of the faculty
in the Fall of 1990.  I have followed the research of Professor Shank
for several years and feel my ideas and interests are similar to his.




%%%At OSU I have concentrated more on AI and been struck by the idea that many AI programs seem to take the knowledge level for granted.  It seems that many relevant, interesting and challenging issues may being swept under the rug.
%%% and began the design and implementation of a frame based object oriented language for home automation.  Even up though graduate school I have continued to expand/revamp this language.  As an RA I designed/implemented a data directed control flow mechanism which is now part of the AI toolset.
%%%In parallel with my discovery of the computer, I began to realize that if I put my mind to it, I could be getting As. So I did and I thought I was a big deal.  
%%%When I was a freshman in college, my calculus teacher, Dr. Klembara, showed me that there was a lot more to learning than getting good grades.  
%%%He motivated me to work to the level of my own ability and taught me a deep respect for analytical discipline.  
%%%Under his influence, my coursework and extra-curricular work became more directed.  
%%%After completing all relevant courses in computer science/mathematics I petitioned for/organized several courses and independent study classes in areas of particular interest to me.
%%%In parallel with my discovery of the computer, I realized that with effort I could overcome my dyslexia and get As.  
%%%Seven years later I wired my family's barn.
%%% in Extensive independent study in this area only confirmed my expectation that artificial intelligence would be an interesting and challenging field for a person with my talents. 


\end{document}

