\documentclass[11pt]{article} 

\usepackage{amsfonts,latexsym,graphics,fullpage}

\begin{document}
\begin{center}
{\bf \Large How to Trust Organizations \\
Provably Unforgeable Self-Testing \\
GHOST \\
}
\end{center}

\section{Introduction}
Big Question: How can large organizations (self)prove that they can be 
trusted?

Several Motivating Scenarios:
\begin{itemize}
\item
How can a police force prove that they don't accept
bribes?
\item
How can the government prove that ballots are properly counted?
\item 
How can the Iraqi police force prove that they don't kill people?
\end{itemize}


The basic question is to design a distributed process wherein honest
and dishonest parties can prove statements of the form 
``we have no more than $\epsilon$-violations of this family 
of tests with probability at least $1-\delta$''.


There are many elements that need to be in place in order for 
this system to work 
\begin{enumerate}
\item Hardware is properly configured.
\item Software is properly configured.
\item Peer-to-Peer algorithms behave as they're supposed to
\item Secure method of executing distributed tests so as to 
accomplish the objective.
\end{enumerate}

We assume that 1,2,3 can be properly solved.  Explain why this
is a feasible assumption.

(4) should be done in a way that the distributed
tests produce a transcript so that there is no conspiracy that
can forge the validation at the end.  Also, the casual observer
and other mutually distrusting parties
can check the transcript to confirm that everyone is behaving
as they should.


\section{Testing Procedure}
What are the tests?  We can answer this question
in specific instances.  For example, to determine if everyone 
reports the truth, we randomly selecting a Perpetrator (P), 
an active officer (A) and an observer (O).  We tell the perpetrator
P to bribe the active officer A.  We ask A,P,O to publish the outcome
of this test.  In the context of voting, we ask many average Joe citizenx
to go to a particular ballot box and count the votes.
If electronic voting, we ask average Joe citizen 
to take the voting machine to a computer
store and test to be sure it works as performed.



\section{Useful Concepts}

\subsection{Strongest Link}
We want the procedures to be as as strong as the strongest link.
In other words, if everyone is being dishonest, except for one
person, we want the procedure to be as good as the honest
person.  This is a difficult requirement especially in light
of crypto where malicious protocols tend to only work when there
is an honest majority.

Note that weakest link is worthless in this context -- we want
to show how to trust organizations.  If the weakest link cannot
be trusted, then the procedure will never produce a useful outcome.

\subsection{Antagonistic Collaborators/Mutually Distrusting Parties}

There are several mutually distrusting 
parties that participate in the protocol.
Dan's examples include the government, Jimmy Carter, ACLU, WACO guys.


\subsection{Time Capsule}
Imagine a collection of mutually distrusting parties
who publish public keys for a particular date $x$ in the future.
The mutually distrusting parties each publish the corresponding
private key on date $x$.

To put data in a time capsule to be released on date $x$, encrypt
the message with the public keys of all the md's for date $x$ (using
onion encryption).
No one will be able to decrypt the message until date x -- even
if all but one mutually distrusting party publishes its private
key before date x.  (Example of strongest link.)


\subsection{Distributed Dice Roll}

We want to roll a distributed dice. Each party contributes a
uniform amount, say between 1,.., 100.  We want to be sure that there
is no way to fake the roll.  We want to be sure that anyone
seeing the roll can determine that the roll was random.

The solution proceeds as follows.  At time 0, the first party
generates a random 128-bit value. At time 1, all other parties
generate/publish a random 128-bit value.  At time 2, the first party computes
the xor of all the random 128-bit values.  If at least one of these
values is random, then the final xor will be random (another example of being
as strong as the strongest link). The first party then uses this
xor value as a key to a random number generator.  The other parties
can verify that the
 first party produces the appropriate random 
number by xor'ing all the 128-bit values and making sure that the
right key was used to the random number generator.


\section{Randomizing the tests}
Do a distributed random roll to pick a random test.
Note, cannot distinguish between test versus non-test.


\section{How to Trust Organizations}



Return to Motivating Scenarios in Introduction .. show how they can 
be trusted.

Explain how multiple mutually-distrusting parties can 
assign a task to an average Joe citizen so that (a) the citizen
is randomly selected (b) the mutually-distrusting parties
don't know who was selected (c) no other citizen, except for
the one that was selected knows who was selected. 
Idea: Use secure multiparty computation to determine if XOR of mutually
distrusting parties random values is equal to the average Joe's citizen's
values.  To assign multiple tasks to multiple citizens, use permutation
array instead of random values.

\section{Conclusions}

\end{document}

